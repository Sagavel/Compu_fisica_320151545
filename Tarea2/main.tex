\documentclass[12pt,a4paper]{article}
\usepackage[utf8]{inputenc}
\usepackage{graphicx}
\usepackage[usenames]{color}
\graphicspath{ {Descargas opera/} }


\title{La imaginación me persigue, pero yo soy más rapido}
\author{Segovia Duran Braulio Azael }               
\date{Fecha de entrega: 11/9/2022}

\begin{document}

\maketitle

\section*{Academia}
\subsection*{Pasado}
\subsubsection*{¿Dé qué escuela vengo?}
Vengo de la prepa 5.
\subsubsection*{¿Quedaba cerca de tu domicilio?}
Me quedaba como a 40 min desde mi casa a la prepa, pero donde si fue dificil fueron mis tiempos de secundaria y primara, algo así como:\\
\includegraphics[width=\textwidth]{mapa}\\
{{\small\it (notese el sarcasmo en la imagen anterior.)}}\\\\
%losé soy muy gracioso, pero fuera de broma mi primaria/secundaria estaban enfrente de mi casa. (y si, muchas veces llegué tarde.)
Mi ruta de transporte eran dos camiones, uno al metro CU y de ahí tomaba una combi que decia literalmente {\sl\fcolorbox{black}{green}{¨prepa 5¨}} , lo puse de color verde porque pues el cartelito de la combi era verde de esos chillones.\\
%hago mi mayor esfuerzo para que se me entienda a mi historia toda fea
Y pues como ya lo había escrito anteriormente, tardaba como 40 min en llegar desde mi casa, hasta entrar a la escuela.


\subsection*{Presente}
\subsubsection*{¿La H. Facultad de Ciencias queda cerca de tu domicilio?}
Quiero pensar que si me queda cerca ya que tardo en total como 45 min, y basandonos en lo que los demas tardan (minimo una hora, y de ahi en adelante) pues me considero afortunado de poder vivir cerca. Mi manera de transpoerte al igual que en la prepa, tenía que tomar un autobus a CU pero en este caso solo tomo el autobus y de ahi me voy caminando hasta la facultad, o a veces agarro el servicio de bici puma.
tardo como 45 minutos pero tardo 30 si ocupo del servicio de bicipuma.\\
\subsubsection*{¿Porqué Fisica?}
Siendo sincero esta fue mi segunda opcion ya que la primera había sido ciencias de la computacion, ya que me gustaría crear apps y diseñar paginas web etc; pero el promedio subío derrepente por lo cual no pude entrar. Dejando de lado eso, a mi me encanta toda la historia y lo relacionado con los cohetes espaciales y la ingeniería aereoespacial y espero poder ver algo similar en esta carrera. 
\section*{Hobbies y pasatiempos}
\subsection*{Descubrir y escuchar música}
No me cconsidero audiófilo obviamente pero si un pequeño fanático del sonido y la música ya que diario descubro y escucho nuevas cacnciones y nuevos artistas. %tambien colecciono vinylos :P 
\subsection*{Reparación y armado de drones} %mi padre trabaja con drones para fotogrametría y escaneo de zonas con daño ambiental 
Reparo, restauro y armo drones para mi padre (y para mi a veces ya que vuelo drones de carreras), me gusta mucho el armado de electrónica porque son como legos... pero con con electricidad ekisde.
\section*{Géneros de música preferidos}
Esto puede ser un poco problemático ya que como dije antes, me encanta escuchar y descubrir música y para intentar dar solo dós generos pues, esta dificil. \\ Pero intentando resumirlo a dos serían algo como hip-hop instrumental y rock espacial (o en pocas palabrass indie rock) 
el hip-hop instrumental lo escucho mucho al caminar, andar en camión o en el auto, no hay una razón en específico porque me gusta, pero me gusta y mucho. 
el rock espacial o indie me gusta escucharlo en mi casa cuando estudio o trabajo (en la reparacion de drones o electrónica en general) 
ninguno de los dos los ocupo para estudiar y que para estudiar ocupo música clasica.
\subsection*{Hip-hop instrumental}
\subsubsection*{Artistas} 
-BadBadNotGood
-Grizzly Bear\\
-Radiohead\\
-The beatles (a veces es hip-hop)\\ 
-Puma Blue 
y etc\\ 
\subsubsection*{Canciones}
-Talk meaning
-Jigsaw falling into place 
-Invierno nuclear, etc \\
\subsection*{Rock espacial (ó indie) }
-Duster\\
-Depresión sonora\\
-Pastel Ghost\\
\subsubsection*{Caniones}
-Capsule losing contact
-Stars will fall 
-Hasta que llegue la muerte\\
\section*{Raíz}
En esta seccion, voy a contar un poco de donde vengo y algunos datos curiosos de mí\\
\subsection*{Dónde nací}
Naci el 21 de marzo en Mérida Yucatan, por lo tanto me considero yucateco.\\ %me sabía un cuento en maya que me lo contó mi bisabuela pero se me olvido:).
Nací el mismo dia que benito juarez, el dia que empieza la primavera y tambien nací el dia en el que el solo refleja una serpiente en la piramide de Kukulcán (Chichen Itza)\\

\includegraphics[width=\textwidth]{kukulcan.jpg}\\

%aquí empieza la parte de los puntos extras, y en este caso yo elegí películas


\section*{Películas}
{\it{(Al igual que con la música, me gusta ver de todo tipo de películas y aqui voy a recomendar algunas que esta buenas.)}}
\subsection*{Películas mexicanaas}
{\bf{ 
-Y tu mamá también (2001)\\
-¡Qué Viva México!" (1932/1979) de Sergei Eisenstein\\
-Los Caifanes (1967)
\subsection*{Peliculas extrangeras}
-El ángel ebrio\\
-Birdman\\ %el director es mexicano pero ps ya ves (Iñarritu) 
-Little miss sunshine}}\subsubsection*{Y este ultimo es un anime que refleja lo bien que puede quedar una animación si se trabaja cuadro por cuadro}

\textcolor{red}{-Redline-}, y tiene una escena que empezó como un corto el cual se llama: \textcolor{yellow}{Yellow Line Race}
%el hecho de que redline este de rojo y yellow line race de amarillo es una muy buena coincidencia pero enserio si buscan ¨redline yellow line race¨ así sale.
\end{document}
